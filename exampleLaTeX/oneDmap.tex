\documentclass{article}

\usepackage{tikz}
\usepackage{pgfplots}
\usepackage{amsmath, amssymb}

\begin{document}

\title{One Dimensional Maps}	
\author{Rajarshi}

\maketitle

\section{Introduction}
Let us look at the {\bf logistic} map.

\begin{align*}
	x_{n+1} &= f(x_n) = rx_n(1 - x_n)
\end{align*}

The context of this relation is lies in population dynamics.
The variable $x$ represents the population
as a fraction of the population
which can be sustained by the system.
Hence the requirement is that

\begin{align*}
	0 \leq x \leq 1
\end{align*}

Given the bounds on $x$ we need require a simialr bound on $f(x)$.
For the problem to be meaningfull we require

\begin{align*}
	0 \leq r \leq 4
\end{align*}

.PY1
def logMap(r, N):
	x = 0.0
	h = 1.0/N
	for i in range(N+1):
		print x, r*x*(1.0-x)
		x += h

def line(xBeg, xEnd, N, m = 1.0, c = 0.0):
	h = (xEnd - xBeg) / N
	x = xBeg
	for i in range(N):
		print x, m*x + c
		x += h
r = 3.5
.PY2

\begin{center}
\begin{tikzpicture}
	\begin{axis}
	[xlabel={$x_n$}, ylabel={$x_{n+1}$},
	xmin=0, xmax=1, ymin=0, ymax=1,
.PY1 logmapset.d
print "ytick={", r/4, "},"
.PY2
	yticklabels={$r/4$}]

		\addplot[blue] table{
.PY1 logmap.d
logMap(r, 50)
.PY2
		};
	\end{axis}
\end{tikzpicture}
\end{center}

\section{Population Evolution}

Value of $r$
\emph{drastically} changes the patterns
we observe in the sequence $x_n$

\begin{center}
.PY1 logGrowth.d
def logGrowth(x, r, N):
	for i in range(1, N):
		print i, x
		x = r*x*(1-x)

paramList = [[0.1, 2.5, 50], [0.2, 3.2, 50], [0.1, 3.7, 50]]
for [x, r, n] in paramList:
	print "\\begin{tikzpicture}\n\\begin{axis}"
	print "[title={$r=", r, "$}]"
	print "\\addplot[only marks] table{"
	logGrowth(x, r, n)
	print "};"
	print "\\end{axis}\n\\end{tikzpicture}\n"
.PY2
\end{center}


\section{Cobweb diagrams}

There is another nice geometric way
of looking at the sequence we are interested in.

\begin{center}
.PY1 cobweb.d
for [x, r, n] in paramList:
	print "\\begin{tikzpicture}\n\\begin{axis}\n"
	print "[xmin=0, xmax=1, ymin=0, ymax=1]"
	print "\\addplot[black] table{"
	line(0.0, 1.0, 50)
	print "};\\addplot[black] table{"
	logMap(r, 50)
	print "};\\addplot table{\n"
	n = 5
	for i in range(n):
		xNext = r*x*(1-x)
		print x, xNext
		print xNext, xNext
		x = xNext
	print "};\\end{axis}\n\\end{tikzpicture}\n"
.PY2
\end{center}
\end{document}
